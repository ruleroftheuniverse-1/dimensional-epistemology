\documentclass[11pt]{article}
\usepackage[utf8]{inputenc}
\usepackage[margin=1in]{geometry}
\usepackage{tikz}
\usepackage{enumitem}
\usepackage{lmodern}
\usepackage{titlesec}
\usepackage{hyperref}

\titleformat{\section}{\large\bfseries}{\thesection.}{0.5em}{}
\titleformat{\subsection}{\normalsize\bfseries}{\thesubsection}{0.5em}{}

\title{\vspace{-2cm}Dimensional Epistemology Worksheet 3\\
\Large{Making a Straightedge with Light and Anything}}
\date{}
\author{}

\begin{document}
\maketitle

\section*{Why (The Situation)}

Kayla wants to teach kids how to draw triangles and rectangles outside—without rulers. They have sticks, rocks, and sunlight. Can they make reliable straight lines without a tool?

Yes. With shadows.

\section*{What (The Goal)}

Use light to project a straight line. Then use that to align objects, trace segments, or construct shapes.

\section*{How (The Method)}

You’ll use two objects and their shadows to define a perfect straight line.

\begin{enumerate}[leftmargin=*, label=\textbf{Step \arabic*.}]
    \item Find a sunny area or strong single light source (like a lamp in a dark room).

    \item Place a vertical object (like a stick, bottle, pencil, or finger) on the ground or table. It will cast a shadow.

    \item Place a second vertical object some distance away. Adjust it until its shadow \textbf{perfectly overlaps} with the first shadow.

    \item The line connecting the two objects is now pointing exactly \textbf{away from the light source}—it’s a straight line in space.

    \item Trace the line between the bases of the two objects using string, chalk, pencil, or more objects.

    \item You’ve made a straightedge with just light and alignment.
\end{enumerate}

\section*{When (The Real World)}

This method works when you have:
\begin{itemize}
    \item No tools, but need straight edges (e.g., chalk lines, planting rows)
    \item Shadow-based measurement (e.g., making sundials or solar calendars)
    \item Visual estimation needs (e.g., checking if shelves or rows are straight)
\end{itemize}

\vspace{1em}
\textbf{Try:}
\begin{itemize}
    \item Use your line to draw a triangle or square on the ground.
    \item Use multiple pairs of objects to extend the line further.
    \item Try this at different times of day and compare light angles.
\end{itemize}

\vspace{1em}
This is geometry, not from abstraction—but from sunlight and stones.

\end{document}
