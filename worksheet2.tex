\documentclass[11pt]{article}
\usepackage[utf8]{inputenc}
\usepackage[margin=1in]{geometry}
\usepackage{tikz}
\usepackage{enumitem}
\usepackage{lmodern}
\usepackage{titlesec}
\usepackage{hyperref}

\titleformat{\section}{\large\bfseries}{\thesection.}{0.5em}{}
\titleformat{\subsection}{\normalsize\bfseries}{\thesubsection}{0.5em}{}

\title{\vspace{-2cm}Dimensional Epistemology Worksheet 2\\
\Large{Finding the Golden Ratio with Just a Page and Patience}}
\date{}
\author{}

\begin{document}
\maketitle

\section*{Why (The Situation)}

Saydee is trying to fold a flyer for her class so that one flap is \textit{pleasingly smaller} than the other. She remembers that book covers, business cards, and many designs use something called the \textbf{golden ratio}—but she doesn’t have a calculator.

What she does have: a single piece of paper, two hands, and a good eye.

\section*{What (The Goal)}

Divide a piece of paper into two parts so that the longer part is roughly 1.618 times the shorter part—using only folds and intuition.

\section*{How (The Method)}

You’ll use geometric relationships hidden in the rectangle itself. Here’s a basic approximation using folds.

\begin{enumerate}[leftmargin=*, label=\textbf{Step \arabic*.}]
    \item Start with a rectangular sheet. Label the corners (mentally or gently with a fingernail) as A (top left), B (top right), C (bottom right), D (bottom left).

    \item Fold corner A diagonally down until it touches the bottom edge somewhere between points D and C. Crease lightly.

    \item Adjust that fold until the edge of the folded-down corner aligns exactly with the side BC. Once it aligns, flatten the crease.

    \item The point where the top edge now intersects the fold line (call it Point E) divides the top edge into two segments.

    \item That division is an approximation of the golden ratio.

    \item Use that length to divide other lengths or create a flap.
\end{enumerate}

\section*{When (The Real World)}

Golden ratios appear in:
\begin{itemize}
    \item Book layouts
    \item Spiral patterns in plants
    \item Logo design and furniture proportions
    \item Wallet-sized prints and flyer folds
\end{itemize}

This method lets you approximate the golden ratio with no measuring. You’ve found a way to unlock harmony through folding.

\vspace{1em}
\textbf{Try:}
\begin{itemize}
    \item Use your golden fold to divide a rope, scarf, or stick.
    \item Compare your fold with a calculator: divide the long length by the short.
    \item Draw a golden rectangle using only your fold and a right angle.
\end{itemize}

\end{document}
