\documentclass[11pt]{article}
\usepackage[utf8]{inputenc}
\usepackage[margin=1in]{geometry}
\usepackage{tikz}
\usepackage{enumitem}
\usepackage{lmodern}
\usepackage{titlesec}
\usepackage{hyperref}

\titleformat{\section}{\large\bfseries}{\thesection.}{0.5em}{}
\titleformat{\subsection}{\normalsize\bfseries}{\thesubsection}{0.5em}{}

\title{\vspace{-2cm}Dimensional Epistemology Worksheet 1\\
\Large{Making a Tenth Fold Without Measuring}}
\date{}
\author{}

\begin{document}
\maketitle

\section*{Why (The Situation)}

Lix has three blank sheets of paper and no ruler. She needs to divide each one into ten equal parts to prepare labels for a preschool activity. But she can’t draw any marks, and the folds have to be clean. She only has her hands, her body, and her furniture.

How can she do it?

\section*{What (The Goal)}

Find a way to fold each sheet into ten equal sections without drawing, marking, or measuring—just using the materials themselves.

\section*{How (The Method)}

This is a visual, projective folding technique. You'll be mapping a tenth segment onto one sheet, then copying it onto the others.

\begin{enumerate}[leftmargin=*, label=\textbf{Step \arabic*.}]
    \item Place Sheet A on a table. Choose a long edge to divide. Label the ends of this edge (mentally or with fingers) as Point A and Point B.
    
    \item Take Sheet B. Line up one of its edges so it touches Point A on Sheet A. Let’s call one of Sheet B’s corners “Point C”.

    \item Take Sheet C. Place one of its corners (Point D) exactly on top of Point C. Rotate or adjust Sheet C so that the edge containing Point D runs \textit{parallel} to the edge AB on Sheet A.

    \item Now, “walk” Sheet C ten times across that AB-parallel edge by flipping end-over-end. Keep track carefully. After the tenth flip, mark the corner as Point E (mentally or with touch).

    \item Now pivot Sheet C around Point E until one of its edges intersects Sheet B somewhere. Call that intersection Point F. Use a fingertip to hold F in place.

    \item Return Sheet C to its original placement with Point D back on Point C and its edge parallel to AB.

    \item Pivot Sheet C again, this time using Point E as an anchor and aligning with Point F. Where this new orientation crosses AB is your division point: 1/10 of the length AB.

    \item Fold Sheet A at this point. Use this fold to replicate on the other sheets.
\end{enumerate}

\section*{When (The Real World)}

This kind of folding isn’t just for labels. Anytime you need to split something evenly—cut wood, fold fabric, prepare doses, plan layouts—you can build fractions using this method.

\vspace{1em}
\textbf{Try:}
\begin{itemize}
    \item Fold a menu or napkin into thirds using similar projection tricks.
    \item Use a piece of string to find 1/2, 1/4, or 1/8 lengths without measuring.
    \item Estimate the golden ratio by folding a page corner toward the opposite edge and adjusting until it “feels” right—then test how close you were!
\end{itemize}

\vspace{1em}
This worksheet was built for learners like Lix. No tools, no shame, just power.

\end{document}
